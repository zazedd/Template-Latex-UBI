\documentclass[chappg, 12pt,a4paper, hidelinks, oneside]{memoir}
% \documentclass[titlepage,12pt,a4paper]{book}

% substituir linha seguinte por 
% \usepackage[english]{babel} 
% se o relatório for escrito na língua inglesa.
\usepackage[portuguese]{babel}

\usepackage[utf8]{inputenc}
\usepackage{makeidx}
\usepackage{xspace}
\usepackage{graphicx,color,times}
\usepackage{fancyhdr}
\usepackage{pxfonts}
\usepackage{times}
\usepackage{mathptm}
\usepackage{amssymb}
\usepackage{amsfonts}
\usepackage{amsmath}
\usepackage{latexsym}
\usepackage[printonlyused]{acronym}
\usepackage{float}
\usepackage{listings}
\usepackage{tocbibind}
\usepackage{natbib}
\usepackage{hyperref}
\usepackage[T1]{fontenc}
\usepackage{titlesec, blindtext, color}
\usepackage{courier}
\usepackage{fix-cm}
\usepackage{fourier}
\usepackage[scaled=.92]{helvet}
\usepackage{geometry}

% \usepackage{glossaries}
% \makeglossaries

% \renewcommand{\ttdefault}{phv}

\pagestyle{fancy}
\renewcommand{\chaptermark}[1]{\markboth{#1}{}}
\renewcommand{\sectionmark}[1]{\markright{\thesection\ #1}}
\fancyhf{} \fancyhead[L,RO]{\bfseries\thepage}
\fancyhead[LO]{\bfseries\rightmark}
\fancyhead[R]{\bfseries\leftmark}
\renewcommand{\headrulewidth}{0.5pt}
\renewcommand{\footrulewidth}{0pt}
\setlength{\headheight}{15.6pt}
\setlength{\marginparsep}{0cm}
\setlength{\marginparwidth}{0cm}
\setlength{\marginparpush}{0cm}
\addtolength{\hoffset}{-1.0cm}
\addtolength{\oddsidemargin}{\evensidemargin}
\addtolength{\oddsidemargin}{0.5cm}
\addtolength{\evensidemargin}{0.5cm}

\titleformat{\chapter}[hang]
    {\Huge\bfseries}{\thechapter\hspace{0.75cm}\hsp\textcolor{gray75}{/}\hspace{0.75cm}}{0pt}{\Huge\bfseries\slshape}
    \titlespacing*{\chapter}{0pt}{-10pt}{20pt}
\titleformat{\section}[hang]
    {\Large\bfseries}{\thesection\hspace{0.5cm}\textcolor{gray75}{|}\hspace{0.5cm}}{0pt}{\Large\bfseries}
\titleformat{\subsection}[hang]
    {\large\bfseries}{\thesubsection\hspace{0.2cm}\textcolor{gray75}{-}\hspace{0.2cm}}{0pt}{\large\bfseries}


\usepackage{fix-cm}
\usepackage{fourier}
\usepackage[scaled=.92]{helvet}
\definecolor{ChapGrey}{rgb}{0.6,0.6,0.6}
\newcommand{\LargeFont}{
  \usefont{\encodingdefault}{\rmdefault}{b}{n}
  \fontsize{60}{80}\selectfont\color{ChapGrey}
  }
\makeatletter
\makechapterstyle{GreyNum}{
  \renewcommand{\chapnamefont}{\large\sffamily\bfseries\itshape}
  \renewcommand{\chapnumfont}{\LargeFont}
  \renewcommand{\chaptitlefont}{\Huge\sffamily\bfseries\itshape}
  \setlength{\beforechapskip}{-20pt}
  \setlength{\midchapskip}{-20pt}
  \setlength{\afterchapskip}{-60pt}
  \renewcommand\chapterheadstart{\vspace*{\beforechapskip}}
  \renewcommand\printchaptername{
  \begin{tabular}{@{}c@{}}
    \chapnamefont \@chapapp\\}
    \renewcommand\chapternamenum{\noalign{\vskip 2ex}}
    \renewcommand\printchapternum{\chapnumfont\thechapter\par}
    \renewcommand\afterchapternum{
  \end{tabular}
  \par\nobreak\vskip\midchapskip}
  \renewcommand\printchapternonum{}
  \renewcommand\printchaptertitle[1]{
  {\chaptitlefont{##1}\par}}
  \renewcommand\afterchaptertitle{\par\nobreak\vskip \afterchapskip}
}
\makeatother
\chapterstyle{GreyNum}

\setcounter{tocdepth}{3}
\setcounter{secnumdepth}{3}
\setsecnumdepth{subsubsection}

\renewcommand{\ttdefault}{lmtt}


% NEW COLORS
\definecolor{gray75}{gray}{0.75}
\definecolor{dark}{gray}{0.25}
\definecolor{lgray}{gray}{0.9}
\definecolor{dkblue}{rgb}{0,0.13,0.4}
\definecolor{dkgreen}{rgb}{0,0.6,0}
\definecolor{gray}{rgb}{0.5,0.5,0.5}
\definecolor{mauve}{rgb}{0.58,0,0.82}

\lstset{ %
  language=C,                    basicstyle=\footnotesize,
  numbers=none,                  numberstyle=\tiny\color{gray}, 
  stepnumber=1,                  numbersep=5pt,
  backgroundcolor=\color{white}, showspaces=false,
  showstringspaces=false,        showtabs=false,
  frame=single,                  rulecolor=\color{black},
  tabsize=2,                     captionpos=b,
  breaklines=true,               breakatwhitespace=false,
  title=\lstname,                keywordstyle=\color{blue},
  commentstyle=\color{dkgreen},  stringstyle=\color{mauve},
  escapeinside={\%*}{*)},        morekeywords={*},
  belowskip=0cm
}

\renewcommand{\lstlistingname}{Excerto de Código}
\renewcommand{\lstlistlistingname}{Lista de Excertos de Código}

\renewcommand{\today}{\day \ifcase \month \or Janeiro\or Fevereiro\or Março\or %
Abril\or Maio\or Junho\or Julho\or Agosto\or Setembro\or Outubro\or Novembro\or %
Dezembro\fi de \number \year} 

\begin{document}


\thispagestyle{empty}
\setcounter{page}{-1}

\begin{center}
\begin{Huge}
\textbf{Universidade da Beira Interior}
\end{Huge}
\end{center}

\begin{center}
\begin{Huge}
Licenciatura em Engenharia Informática
\end{Huge}
\end{center}

\vspace{0,07cm}
\begin{figure}[!htb]
\centering
\includegraphics[width=191pt]{ubi-fe-di.png}
\end{figure}

\vspace{0.5cm}
\begin{center}
\begin{Large}
\textbf{\emph{NOME PROJETO}}
\end{Large}
\end{center}


\vspace{0.30cm}
\begin{center}
\begin{normalsize}
\begin{large}
Elaborado por:
\end{large}
\end{normalsize}
\end{center}

\vspace{0.15cm}
\begin{center}
\begin{large}
\textbf{Nome do Aluno}
\end{large}
\end{center}

\vspace{0.35cm}
\begin{center}
\begin{normalsize}
\begin{large}
Orientador:
\end{large}
\end{normalsize}
\end{center}

\vspace{0.15cm}
\begin{center}
\begin{large}
\textbf{Professor Doutor ...}
\end{large}
\end{center}

\vspace{0.35cm}
\begin{center}
\begin{normalsize}
\begin{large}
Unidade Curricular:
\end{large}
\end{normalsize}
\end{center}

\vspace{0.15cm}
\begin{center}
\begin{large}
\textbf{UC}
\end{large}
\end{center}

\vspace{0.05cm}
\begin{center}
\begin{normalsize}
\today, Covilhã
\end{normalsize}
\end{center}


\clearpage{\thispagestyle{empty}}
\phantom{p. 1}
\clearpage

\frontmatter

\chapter*{Agradecimentos}
\label{chap:ack}
\vspace{0.7cm}

Agradece aqui a quem te apetecer.

\chapter*{Resumo}
\label{chap:res}
\vspace{0.7cm}

Resumo breve aqui.

\chapter*{Palavras-Chave}
\label{chap:res}
\vspace{0.7cm}

\begin{itemize}
    \item Palavras
\end{itemize}


\setlength{\beforechapskip}{0pt}
\setlength{\midchapskip}{0pt}
\setlength{\afterchapskip}{0pt}
{\footnotesize\tableofcontents}
\setlength{\beforechapskip}{-20pt}
\setlength{\midchapskip}{-20pt}
\setlength{\afterchapskip}{-60pt}

\chapter*{Acrónimos}
\label{chap:acro}

% #   ATENÇÃO
% A lista de acrónimods deve ser ordenada alfanumericamente.
% Estrangeirismos devem ser realçados em itálico.
% Se o relatório for escrito em Inglês, uma palavra portuguesa é um estrangeirismo.

% O maior acrónimo deve ser colocado neste ponto.
%               vvvvvv
\begin{acronym}[UBI]

  \acro{UBI}{Universidade da Beira Interior}

\end{acronym}

\listoffigures
\clearpage{\thispagestyle{empty}}
\listoftables

% #   ATENÇÃO
% Se existirem trechos de código, descomentar as seguintes linhas
% \clearpage{\thispagestyle{empty}\cleardoublepage}
% \lstlistoflistings

% \clearpage{\pagestyle{empty}\cleardoublepage}
% \include{glossario}

\phantom{p. 1}
\clearpage
\thispagestyle{empty}
\phantom{p. 2}
\clearpage

\mainmatter
\acresetall
\chapter{Introdução}
\label{chap:intro}

\section{Enquadramento}
\label{sec:amb} % CADA SECÇÃO DEVE TER UM LABEL
% CADA FIGURA DEVE TER UM LABEL
% CADA TABELA DEVE TER UM LABEL

Lorem Ipsum

\section{Motivação}
Motivação \ac{UBI} 
\label{sec:mot}

\section{Objetivos}
\label{sec:obj}

\section{Organização do Documento}
\label{sec:organ}
% !POR EXEMPLO!
De modo a refletir o trabalho que foi feito, este documento encontra-se estruturado da seguinte forma:
\begin{enumerate}
\item O primeiro capítulo -- \textbf{Introdução} -- apresenta o projeto, a motivação para a sua escolha, o enquadramento para o mesmo, os seus objetivos e a respetiva organização do documento.
\item O segundo capítulo -- \textbf{Tecnologias Utilizadas} -- descreve os conceitos mais importantes no âmbito deste projeto, bem como as tecnologias utilizadas durante do desenvolvimento da aplicação.
\item ...
\end{enumerate}

\clearpage{\thispagestyle{empty}\cleardoublepage}
\chapter{Ch2}
% OU \chapter{Trabalhos Relacionados}
% OU \chapter{Engenharia de Software}
% OU \chapter{Tecnologias e Ferramentas Utilizadas}
\label{chap:estado-da-arte}

\section{Introdução}
\label{chap2:sec:intro}

Lorem Ipsum

\section{Topico 1}
\label{chap2:sec:citacoes}

Lorem Ipsum

\section{Topico 2}
\label{chap2:sec:...}

Lorem Ipsum

\section{Topico 3}
\label{chap2:sec:concs}

Lorem Ipsum

\section{Conclusão}
\label{chap2:sec:concs}

Lorem Ipsum
\clearpage{\thispagestyle{empty}\cleardoublepage}
\chapter{Ch3}
% OU \chapter{Trabalhos Relacionados}
% OU \chapter{Engenharia de Software}
% OU \chapter{Tecnologias e Ferramentas Utilizadas}
\label{chap:tecno-ferra}

\section{Introdução}
\label{chap3:sec:intro}

Lorem Ipsum

\section{Topico 1}
\label{chap3:sec:...}

Lorem Ipsum
\section{Topico 2}
\label{chap3:sec:concs}

Lorem Ipsum

\section{Topico 3}
\label{chap3:sec:concs}

Lorem Ipsum

\section{Conclusão}
\label{chap3:sec:concs}

Lorem Ipsum
\clearpage{\thispagestyle{empty}\cleardoublepage}
\chapter{Ch4}
% Os titulos dados aos capítulos são meros exemplos. Cada relatório deve adequar-se ao projeto desenvolvido.
\label{chap:imp-test}

\section{Topico 1}
\label{chap4:sec:intro}

Lorem Ipsum

\subsection{SubTopico 1}

Lorem Ipsum

\section{Topico 2}
\label{chap4:sec:...}

Lorem Ipsum

\section{Topico 3}
\label{chap4:sec:concs}

Lorem Ipsum

\subsection{SubTopico 2}

Lorem Ipsum

\subsection{SubTopico 3}

Lorem Ipsum

\section{Conclusão}
\label{chap4:sec:concs}

Lorem Ipsum

\clearpage{\thispagestyle{empty}\cleardoublepage}
\chapter{Conclusões e Trabalho Futuro}
\label{chap:conc-trab-futuro}

\section{Conclusões Principais}
\label{sec:conc-princ}

Esta secção contém a resposta à questão: \\
\emph{Quais foram as conclusões princípais a que o(a) aluno(a) chegou no fim deste trabalho?}

Indicar o que foi conseguido.
Indicar o que não foi conseguido. Indicar a(s) razão(ões).

\section{Trabalho Futuro}
\label{sec:trab-futuro}

Esta secção responde a questões como:\\
\emph{O que é que ficou por fazer, e porque?}\\
\emph{O que é que seria interessante fazer, mas não foi feito por não ser exatamente o objetivo deste trabalho?}\\
\emph{Em que outros casos ou situações ou cenários -- que não foram estudados no contexto deste projeto por não ser seu objetivo -- é que o trabalho aqui descrito pode ter aplicações interessantes e porque?}
\clearpage{\thispagestyle{empty}\cleardoublepage}

\backmatter

\phantom{p. 1}
\clearpage

\chapter*{Epílogo}

Lorem Ipsum


\bibliographystyle{unsrt}
\bibliography{bibliografia}

\chapter*{Anexos}

Colocar aqui o que for acessório para a leitura do trabalho (não desenvolvido pelos autores).

\appendix
\include{apendice1}
\clearpage{\pagestyle{empty}\cleardoublepage}
\include{continuacao}
\clearpage{\pagestyle{empty}\cleardoublepage}
\include{apendice2}
\clearpage{\pagestyle{empty}\cleardoublepage}
\include{apendice3}
\clearpage{\pagestyle{empty}\cleardoublepage}

\end{document}